\usepackage{booktabs}
\usepackage{hyperref}
\usepackage{graphicx}
\usepackage{subfig}
%\usepackage{subcaption}
\usepackage[inline]{enumitem}   
\usepackage{ltablex}
\usepackage{color}	 
\usepackage{amsmath,scalerel}
\usepackage{amsfonts}
\usepackage{url}
\usepackage{balance}
\usepackage{bm}
\usepackage{bbm} %for indicator function

%\usepackage{notes}
\usepackage{todonotes}
\usepackage{tabularx}
\usepackage{multirow}
\usepackage[ruled, linesnumbered, titlenumbered]{algorithm2e}
%\usepackage{titlesec}
%\usepackage{python} 
\let\proof\relax
\let\endproof\relax
\usepackage{amsthm}

\newcommand{\rapcomment}[1]{\footnote{RAP: #1}}
%\newcommand{\rapcomment}[1]{}

%\newcommand{\zzcomment}[1]{\footnote{ZZ: #1}}
\newcommand{\zzcomment}[1]{}

\DeclareMathOperator*{\argmin}{argmin}
\DeclareMathOperator*{\argmax}{argmax}
\newcommand{\dataset}{\mathcal{D}}
\newcommand{\trainset}{\mathcal{R}}
\newcommand{\testset}{\mathcal{T}}
\newcommand{\trainsetLT}{\mathcal{R}^{L}}
\newcommand{\testsetLT}{\mathcal{T}^{L}}

\newcommand{\usersofIteminTrainset}{\mathcal{U}_{i}^{\trainset}}
\newcommand{\usersofIteminTestset}{\mathcal{U}_{i}^{\testset}}
\newcommand{\itemsofUserinTrainset}{\mathcal{I}_{u}^{\trainset}}
\newcommand{\itemsofUserinTestset}{\mathcal{I}_{u}^{\testset}}
\newcommand{\itemsinTrainset}{\mathcal{I}^\trainset}

\newcommand{\simpleRisk}{\theta_u^{S}}
\newcommand{\LTRisk}{\theta_u^{L}}
\newcommand{\heuristicRisk}{\theta_u^{H}}
\newcommand{\tfidfRisk}{\theta_u^{T}}
\newcommand{\size}{N}



%%%%%%%%%%%%%%%%%%%%%%%%%%%%%%%%%%%%%%%%%%%%%%%%%%%%%%%%%%%%%%%%%%%%%%%%%%%%%%%%%%%%%%%%%5
%%%%%%%%%%%%%%%%%%%%%%%%%%%%%%%%%%%%%%%%%%%%%%%%%%%%%%%%%%%%%%%%%%%%%%%%%%%%%%%%%%%%%%%%%%%%%
\newtheorem{defin}{Definition}
\newtheorem{prblm}{Problem}
\newtheorem{chal}{Challenge}
\newtheorem{ex}{Example}
\newtheorem{thm}{Theorem}[section]
\newtheorem{lem}{Lemma}[section]
\newtheorem{corol}{Corollary}[section]
\newtheorem{clam}{Claim}[section]

\newenvironment{defn}{\begin{defin}\begin{rm}}
  {{\hfill$\Box$}\end{rm}\end{defin}}

%define how to show the previously named theorems                                                      
\newenvironment{theorem}{\begin{thm} \nopagebreak}{\end{thm}}
\newenvironment{claim}{\begin{clam} \nopagebreak}{\end{clam}}
\newenvironment{lemma}{\begin{lem} \nopagebreak}{\end{lem}}
%\renewenvironment{proof}{\noindent {\bf Proof } \nopagebreak
%  \begin{normalsize}}{\end{normalsize}{\hfill$\Box$}}
\newenvironment{definition}{\begin{defin}\begin{rm}}
  {{\hfill}\end{rm}\end{defin}}
\newenvironment{problem}{\begin{prblm}\begin{rm}}
  {{\hfill}\end{rm}\end{prblm}}
\newenvironment{challenge}{\begin{chal}\begin{rm}}
  {{\hfill$\Box$}\end{rm}\end{chal}}
\newenvironment{example}{\begin{ex} \nopagebreak                                                       
  \begin{rm}}{{\hfill}\end{rm}\end{ex}}
\newenvironment{corollary}{\begin{corol} \nopagebreak}{\end{corol}}


\numberwithin{equation}{section}

% MATH -----------------------------------------------------------
\newcommand{\norm}[1]{\left\Vert#1\right\Vert}
\newcommand{\abs}[1]{\left\vert#1\right\vert}
\newcommand{\set}[1]{\left\{#1\right\}}
\newcommand{\Real}{\mathbb R}
\newcommand{\eps}{\varepsilon}
\newcommand{\To}{\longrightarrow}
\newcommand{\BX}{\mathbf{B}(X)}
\newcommand{\A}{\mathcal{A}}

%\usepackage[bookmarks,bookmarksnumbered,%
%    citebordercolor={0.8 0.8 0.8},filebordercolor={0.8 0.8 0.8},%
%    linkbordercolor={0.8 0.8 0.8},%
%    pagebackref%
%    ]{hyperref}
%\renewcommand{\sectionautorefname}{Section}
%\renewcommand{\subsectionautorefname}{Section}
%\renewcommand{\subsubsectionautorefname}{Section}

%\newcommand{\autoref}{\ref}
\newcommand{\obj}{compound}
\newcommand{\objs}{compounds}
\newcommand{\ele}{component}
\newcommand{\eles}{components}
\newcommand{\qnode}{$q$-$node$}
\newcommand{\dnode}{$d$-$node$}
\newcommand{\anode}{$a$-$node$}
\newcommand{\theconfname}{PVLDB}
%\newcommand{\sys}{HetIS}
%\newcommand{\syss}{HetISs}
\newcommand{\sys}{heterogeneous information system}
\newcommand{\syss}{heterogeneous information systems}

\newif\ifproposal
%\proposaltrue
\proposalfalse

\newif\iffullpaper
%\fullpaperfalse %if we were doing the full version, change to "\fullpapertrue"
\fullpapertrue
\newif\ifnotfullpaper
%\notfullpaperfalse
\notfullpapertrue %if we were doing the full version, change to "\fullpapertrue"

\newif\ifextraparts % should always be false. check with grep -nr 'extraparts' to see where it is used
\extrapartsfalse
%\extrapartstrue

\newif\ifbootstrapexp
\bootstrapexptrue
%the following help to save space:
\newcommand{\eat}[1]{} % the eat command just deletes text
\newenvironment{changemargin} [2]{\begin{list}{}{
         \setlength{\topsep}{0pt}\setlength{\leftmargin}{0pt}
         \setlength{\rightmargin}{0pt}
         \setlength{\listparindent}{\parindent}
         \setlength{\itemindent}{\parindent}
         \setlength{\parsep}{0pt plus 1pt}
         \addtolength{\leftmargin}{#1}\addtolength{\rightmargin}{#2}
         }\item }{\end{list}}

\newenvironment{myitemize} %makes a shorter version of itemize
  {
    \begin{changemargin}{-8pt}{-0cm}
    \vspace{-13pt}
    \hspace{-8pt}
    \begin{itemize}
    \setlength{\itemsep}{-1pt}
  }
  {
    \end{itemize}
    \end{changemargin}
  }

\newenvironment{myenumerate} %a shorter version of enumerate
  {
    \begin{changemargin}{-8pt}{-0cm}
    \vspace{-13pt}
    \hspace{-8pt}
    \begin{enumerate}
    \setlength{\itemsep}{-1pt}
  }
  {
    \end{enumerate}
    \end{changemargin}
  }
  
\renewcommand{\sectionautorefname}{\S}
%\newcommand{\norm}[1]{\lVert#1\rVert}


\newlength\mystoreparindent
\newenvironment{myparindent}[1]{%
\setlength{\mystoreparindent}{\the\parindent}
\setlength{\parindent}{#1}
}{%
\setlength{\parindent}{\mystoreparindent}
}
